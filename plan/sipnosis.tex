\section*{Argumento}
\begin{itemize}
\item La novela explora las consecuencias del desarrollo de la tecnología de regeneración celular (RC) que permite rejuvenecer a las personas en ciclos aproximados de 30 años. La RC se aplica alrededor de los 60 y devuelve al sujeto a los 30 años en todos los aspectos fisiológicos, incluyendo apariencia. Es decir, un viejo de 60 sale del tratamiento con el aspecto físico y la salud de un joven de 30 (incluyendo corrección de posibles enfermedades, como el cáncer, etc.). 
\item Sin embargo, la aplicación de la RC a las neuronas cerebrales es un proceso muy complejo. Requiere la activación de sinapsis silenciosas (SS) que reemplazan las sinapsis que se destruyen durante el proceso de RC entre neuronas. El resultado del \gl reseteo neuronal\gr~ (RN) es impredecible hasta cierto punto. Se pierden siempre una parte (a menudo muy grande) de los recuerdos y la personalidad del individuo se altera fuertemente. Un individuo recién salido del RN no se reconoce a sí mismo como el individuo anterior y no reconoce como propios ni los recuerdos ni los sentimientos de este. 
\item La versión más \gl suave\gr~del fenómeno, que se da en un 70\% de los casos es que el individuo recuerde las memorias y sentimientos de su \gl yo\gr~anterior como \gl ajenas\gr~pero \gl familiares\gr~, algo así como si su yo anterior fuera el protagonista de un libro que ha leído y cuyas andanzas le resultan familiares y en general \glr{agradables}. En estos casos, es posible \glr{recuperar}~la personalidad mediante un proceso de hipnosis, que \glr{restaura}~el yo, o, en realidad, la ilusión del yo. El individuo, después de la hipnosis tiene un juego de recuerdos \glr{equivalente}~(pero no idéntico) a los del individuo original, recuerda los grandes trazos de la historia de su vida y todos los acontecimientos relevantes, pero la carga emocional es diferente a menudo, anécdotas sin importancia (para el yo anterior) se amplifican y otras pierden significado. En cuanto a los sentimientos, se produce la ilusión de continuidad, pero en general están \glr{adormecidos}, es decir el individuo siente sus sentimientos con menor intensidad. Después de reincorporarse a su vida habitual, sin embargo, esa intensidad se recupera (o al menos eso parece). En realidad, los sentimientos nunca se recuperan, pero la mayoría de los individuos aprenden a creerse la simulación de estos, al menos provisionalmente. Es un secreto a voces que los matrimonios no duran después de la regeneración, los vínculos familiares se rompen o debilitan, las relaciones de amistad se resienten, etc. De hecho, mencionar este aspecto es tabú en los círculos sociales que usan RC (y en la empresa que lo gestiona). 
\item En el otro 30\% de los casos, se desarrollan todo tipo de patologías. Desde pérdidas irrecuperables de memoria, hasta memorias desplazadas (el individuo evoca un recuerdo distante años o décadas como muy reciente), incluyendo memorias falsas (hechos que han afectado a la persona, aunque no le ocurrieran a él y que recuerda como propios). En lo que se refiera a sentimientos, a menudo se da un rechazo implícito e incontrolable hacia uno o más seres queridos, incluyendo síndrome de Capgras (la persona querida se percibe como un impostor) y otros trastornos similares. 
\item El resultado es que al proceso conocido como {\em Renova} (Regeneración Neuronal, Orgánica y Vascular Avanzada) es, en realidad, un trato faustiano (pacto con el diablo) cargado de aristas morales. En la novela, diferentes personajes la critican desde diferentes ángulos.
\item La crítica más aguda (Claudia, Diego) es que Renova, en la práctica, aniquila al individuo que se somete al proceso y lo sustituye por otro parecido, una especie de clon, o aproximación, una copia imperfecta. La tesis aquí es que la persona no es otra cosa que un proceso dinámico en el que intervienen de manera muy compleja recuerdos, vivencias, sentimientos, etc. El RN destruye en todos los casos una parte tan significativa de ese material, que la persona que emerge es diferente (por lo tanto la persona anterior ha desaparecido). De hecho, todo el que emerge de Renova se siente así, sólo recupera \glr{el contacto} con su yo anterior a partir de la hipnosis. Desde ese punto de vista, la jugada diabólica es que el individuo que se somete a Renova, realmente se está suicidando a cambio de la ilusión de continuidad y, eso sí, escapar de la vejez. En ese sentido, cada vez que bebe de la fuente de la eterna juventud (Panacea) el individuo renuncia a su ser a cambio de rejuvenecer. Vende, literalmente, su alma al diablo. 
\item La segunda crítica es que el proceso sólo está al alcance de los vips e introduce la mayor y más insufrible desigualdad entre seres humanos. Unos tienen derecho a la eterna juventud y los otros no. Los proles trabajan para que los vips tengan acceso a la vida eterna. 
\item La tercera crítica, que viene incluso de sectores vip es que Renova genera una sociedad estancada y sin esperanza. Los viejos se \glr{renuevan}~una y otra vez bloqueando el acceso a los jóvenes. Las relaciones personales (en particular las familias, tanto las parejas como las relaciones padre-hijo) se desintegran en el proceso y el mundo vip tiene a convertirse en un erial de egoísmo y soledad. De ahí el título, \glr{La Tierra Baldía}.  
\end{itemize}

\section*{Construcción}
\begin{itemize}
\item La novela incluye dos historias principales. La de Alan y Quim y la de Alan y Claudia. Además hay varias historias secundarias. La organización desordena la narración repartiéndola entre varios puntos de vista y moviéndola de forma no lineal en el tiempo. 
\end{itemize}


\section*{Alan y Quim}

\begin{itemize}
\item Alan Turner nace en 2001, es decir, el año de los atentado del 11S, que, en el simbolismo de La Tierra Baldía, supone un equivalente al principio de la guerra. 
\item   Alan es el único hijo de una familia vip norteamericana (cuya residencia principal está en San Francisco).  Los Turner poseen una gran mansión en Alba, un pueblo de la costa mediterránea, que visitan cada verano durante dos meses. La madre de Alan (Estela), es la hija de los marqueses de Chicharri, una familia de nobles más o menos arruinados que residen en el castillo de Alba. Estela, bella, fría y calculadora, viaja a USA para estudiar, allí conoce a Benjamín (Ben), hijo de la riquísima familia Turner y a su vez exitoso hombre de negocios en el mundo de la farmacia.  Estela se casa con él, adquieren el castillo de Alba (a sus padres arruinados) y toma la costumbre de viajar los veranos, para que Alan aprenda español y para no perder el contacto con su familia (Estela tiene su propia agenda, explotar al máximo el dinero de su marido y ampliar su influencia sobre él, en USA tiene que competir con su familia política, en Alba lo tiene más a su disposición).  
\item El castillo de Alba no tiene ama de llaves sino \glr{manager}, Miguel Montañés, antiguo patrón de pesca, que además de gestionar la mansión (donde siguen viviendo los insoportables marqueses) se ocupa del yate de los Turner que es la afición principal de Ben. Quim, el hijo de Miguel, acompaña a menudo a su padre y así se convierte en el único amigo de Alan. Alan perfecciona su español con Quim pero este aprende a hablar inglés pasmosamente bien, dando indicios de su genialidad. 
\item La historia de su niñez tiene que reflejar la tensión entre dos mundos. Quim, en casa de Alan, escucha protestar a su abuela a la que disgusta tener por la casa a todas horas al hijo del criado. Se traga la humillación y no dice nada, pero después vendrá la escena de \glr{sal por esa puerta}. Otras historias se suceden, a medida que pasan los veranos. Cuando vienen los primos de US a la casa de Alba, Quim es instruido para  que no aparezca durante la semana que dura la visita (como siempre, Alan no lo sabe) y la respuesta es el vacío en el pueblo a Alan. Alan toma clases de piano y enseña a Quim durante dos meses, pero cuando se va la familia de Quim no tiene interés o recursos para pagarle clases. El verano siguiente es el ajedrez, Alan da clases con un maestro pero no tiene mucho talento. Quim si lo tiene, aprende online, destroza a Alan y luego se niega a jugar más con él (es su venganza por la música). 
\item De su niñez y de su torturada amistad, se va a fraguar el sentimiento de rechazo de sus respectivos mundos, la lucha de clases. En Quim, esto le lleva a querer acabar con los privilegios vip, en Alan, a la convicción que hay que reducir la desigualdad. Quim, que odia visceralmente a los vips, representados por los abuelos que le llamaron el hijo del criado pero también por la actitud falsamente condescendiente de Estela y Ben, ve a Alan como un \glr{buenista}~que en el fondo no va a hacer nada por los proles, un hipócrita.  
\item El verano que ambos cumplen 16 años, los Turner no pueden viajar a Alba, pero envían al hijo. Miguel tiene el encargo de organizar un crucero de vips, recogiendo gente en varios puertos mediterráneos, todo vips parte de las conexiones de Ben y Estela. Propone llevar a su hijo (Quim) y a la prima de su hijo (Ariadna) como tripulación, Alan insiste para que lo enrolen también a él \glr{de incógnito}. Ben, cuya actitud es diferente a la de los Chicharri (va de rico \glr{cool}~y colega) accede y durante dos meses se desarrolla la amistad entre los tres. Alan se enamora de Ariadna, está convencido de que ella le corresponde y se queda patidifuso cuando la sorprende con Quim. Lo han engañado todo el verano. Las razones de Quim (que Ariadna secunda) es que no tenían opción, Quim es el hijo del amo y se habría tomado a mal el rechazo, pensaban ponerle al corriente al terminar el viaje. Alan se lleva un batacazo tremendo pero aguanta el tipo, ya que no tiene nadie más que ellos dos. 
\item Ariadna tiene un talento inusitado para el arte. El verano siguiente, cuando llega la familia, Alan organiza una exposición de los cuadros de Ariadna que impresionan a Ben. Ese verano, Quim no está. Le han dado una beca en Oxford para ampliar estudios de bioquímica, queda claro que Quim es un genio. Ariadna le confiesa un secreto oscuro que no lo es tanto para Alan. Quim es homsexual. Alan lo sabe, porque tuvo una experiencia frustrada con él. Ariadna ha intentado asimilarlo, sin éxito, han roto. Al final, se lía con Alan. 
\item Alan aplica a Stanford y anima a Quim a que haga lo mismo. Alan es aceptado, Quim desestimado, a pesar de las inmejorables recomendaciones. Ariadna es admitida con una beca en la facultad de arte, por influencia de los Turner. 
\item Siguen los años de California. Alan empieza a trabajar en las SS, Ariadna tiene éxito y empieza a hacer cuadros-holos. Vida \glr{bohemia de lujo}~ en SF. Se casan. Apenas saben nada de Quelo, que ha continuado sus estudios en Oxford. 
\item En 2030, inesperadamente, Quim y Alan son co-recipientes del premio Breakthrough the biología, uno por sus trabajos sobre RC y el otro por sus estudios de SS. Ariadna viaja con Alan y los tres amigos se reencuentran. Quim está muy estropeado, bebe mucho, toma drogas y su vida sexual es un desastre. Ha contraído SIDA (pero se ha curado, ya existe la cura) etc. Parece llevar una vida suicida, de la que sólo le rescata su obsesión por la ciencia.  Alan se lo lleva a California, Ariadna y él lo cuidan mientras pasa por detox y se recupera. Cuando vuelve a Inglaterra, Ariadna le confiesa que se ha acostado con él. Era un \glr{asunto pendiente}. Alan lo acepta, pero se lo come el rencor y cuando Quim intenta mantener la relación lo rechaza. 
\item A lo largo de las siguientes décadas Alan desarrolla la téncia del \glr{reseteo neuronal}\footnote{\url{https://www.economist.com/science-and-technology/2022/12/07/how-adult-brains-learn-the-new-without-forgetting-the-old?utm_content=ed-picks-article-link-3&etear=nl_special_3&utm_campaign=a.coronavirus-special-edition&utm_medium=email.internal-newsletter.np&utm_source=salesforce-marketing-cloud&utm_term=7/22/2023&utm_id=1692919}} y Quim perfecciona la regeneración celular\footnote{\url{https://scitechdaily.com/age-reversal-breakthrough-harvard-mit-discovery-could-enable-whole-body-rejuvenation/?expand_article=1}}. Alan gana el premio Nobel en 2050. Mientras tanto, Quim ha fundado Panacea y tiene éxito en regeneración celular, revirtiendo el proceso de envejecimiento... pero el límite está en el cerebro, las neuronas no se renueven y de hecho las drogas que usan en Panacea parece acelerar la degeneración neuronal, de tal manera que se desarrolla pronto demencia senil. Aun así, se ha hecho millonario, Panacea es la compañía más valorada de todo el mercado. 
\item Quim es invitado a la entrega del Premio Nobel (no por Alan, directamente por el comité Nobel) y le propone a Alan que unan esfuerzos. Pero Alan no lo tiene claro y Ariadna menos todavía. Panacea sólo está al alcance de los vips, lo que va en contra de los ideales de Alan y Ariadna. Quim parece pasar de todo en su obsesión con desarrollar Renova. Pronto se revela su motivación. Renova garantiza una aproximación casi perfecta a la inmortalidad (aún no saben nada de efectos secundarios), los tres ya tienen 50 años y él no quiere morirse en otros 30 o 40 ni quiere envejecer. Los primeros beneficiados de su tecnología serían ellos. Alan se siente tentado. No así Ariadna, que opina que es mejor vivir una buena vida que muchas malas.  
\item No obstante las cosas cambian cinco años después. Ariadna desarrolla un glioblastoma que sólo puede curar un tratamiento agresivo que a su vez implica combinar las técnicas de regeneración celular y aplicar por primera vez la técnica de sinapsis silenciosas. El resultado es que Ariadna se cura, pero pierde la memoria y desarrolla síndrome de Capgras (Ariadna en Naxos). 
\item Alan invierte un lustro en desarrollar las técnicas de recuperación de memoria y restauración hipnótica tratando de rescatar a Ariadna, pero el cáncer recurre y muere sin reconocerle. Tiene 60 años, está solo y considera que no tiene nada que perder, sabe que las técnicas de REM (recuperación memoria) van a funcionar y no quiere seguir envejeciendo con el pesar de haber perdido a Ariadna. Decide hacer de conejillo de indias de su propia técnica. Su plan es dedicar su siguiente ciclo a conseguir que Renova sea accesible a todo el mundo.  
\item La regeneración es un éxito. Quim se dedica a Panacea y Alan a mejorar la hipnosis. Vuelve a Alba al principio de su segundo ciclo (2095) interrumpiendo la hipnosis después de unas pocas sesiones, es todavía un tipo "diferente". Conoce a Nerea y se casa con ella, pero lo localizan y le obligan a completar la hipnosis (quizás legalmente, un rapto legal autorizado por su yo anterior). Cuando sale ya no quiere a Nerea que se deprime y se suicida siendo Claudia un bebé. 
\item Quim, al despertarse en el segundo ciclo ha perdido el interés por ser un vip y recuperado el deseo de igualdad, etc. Desaparece y monta el movimiento anti-Panacea, mientras que Alan se hace cargo de la compañía. Claudia crece obsesionada con ayudar a su padre a "arreglar" el problema de la hipnosis hasta que sus propios estudios demuestran que el problema no tiene arreglo. Hay una parte esencial de la personalidad que se desvanece en el proceso y la persona resultante siempre es un parche, un frankenstein. 
\item Claudia se refugia en Alba donde Quim le cuenta la historia de sus vidas. 
\item Quim la alista para combatir legalmente a Panacea. 
\item Alan viaja a Alba, de nuevo a medio proceso hipnótico, pero esta vez ha decretado que no lo fuercen. Allí se encuentra con Claudia y con Quim. Final de la historia de amor "imposible". 
\end{itemize}
